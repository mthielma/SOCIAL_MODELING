\documentclass[11pt]{article}
\usepackage{geometry}                
\geometry{letterpaper}                   

\usepackage{graphicx}
\usepackage{amssymb}
\usepackage{epstopdf}
\usepackage{natbib}
\usepackage{amssymb, amsmath}
\DeclareGraphicsRule{.tif}{png}{.png}{`convert #1 `dirname #1`/`basename #1 .tif`.png}

%\title{Title}
%\author{Name 1, Name 2}
%\date{date} 

\begin{document}



\thispagestyle{empty}

\begin{center}
\includegraphics[width=5cm]{ETHlogo.eps}

\bigskip


\bigskip


\bigskip


\LARGE{ 	Lecture with Computer Exercises:\\ }
\LARGE{ Modelling and Simulating Social Systems with MATLAB\\}

\bigskip

\bigskip

\small{Project Report}\\

\bigskip

\bigskip

\bigskip

\bigskip


\begin{tabular}{|c|}
\hline
\\
\textbf{\LARGE{Evacuation bottlenecks in flooding events}}\\
\textbf{\LARGE{}}\\
\\
\hline
\end{tabular}
\bigskip

\bigskip

\bigskip

\LARGE{Fabio Crameri \& Marcel Thielmann}



\bigskip

\bigskip

\bigskip

\bigskip

\bigskip

\bigskip

\bigskip

\bigskip

Zurich\\
December 2011\\

\end{center}



\newpage

%%%%%%%%%%%%%%%%%%%%%%%%%%%%%%%%%%%%%%%%%%%%%%%%%

\newpage
\section*{Agreement for free-download}
\bigskip


\bigskip


\large We hereby agree to make our source code for this project freely available for download from the web pages of the SOMS chair. Furthermore, we assure that all source code is written by ourselves and is not violating any copyright restrictions.
Several external packages are used that are freely available at Matlab FileExchange under a BSU license. 
\begin{center}

\bigskip


\bigskip


\begin{tabular}{@{}p{3.3cm}@{}p{6cm}@{}@{}p{6cm}@{}}
\begin{minipage}{3cm}

\end{minipage}
&
\begin{minipage}{6cm}
\vspace{2mm} \large Fabio Crameri

 %\vspace{\baselineskip}

\end{minipage}
&
\begin{minipage}{6cm}

\vspace{2mm} \large Marcel Thielmann

\end{minipage}
\end{tabular}


\end{center}
\newpage

%%%%%%%%%%%%%%%%%%%%%%%%%%%%%%%%%%%%%%%



% IMPORTANT
% you MUST include the ETH declaration of originality here; it is available for download on the course website or at http://www.ethz.ch/faculty/exams/plagiarism/index_EN; it can be printed as pdf and should be filled out in handwriting


%%%%%%%%%% Table of content %%%%%%%%%%%%%%%%%

\tableofcontents

\newpage

%%%%%%%%%%%%%%%%%%%%%%%%%%%%%%%%%%%%%%%



\section{Abstract}

\section{Individual contributions}

\section{Introduction and Motivations}

\section{Description of the Model}

\subsection{Social forces}
\subsubsection{Repulsive forces}
\subsubsection{Attractive forces}
\subsection{Physical forces}
\subsection{Walking speed}
\subsection{Flooding}

\section{Implementation}
This section is meant to both give an overview of the methods used in this code as well as to provide a documentation of the code. Therefore, some details are mentioned here that might not be crucial for any code that simulates pedestrian dynamics, but are needed in our implementation. 
\subsection{Initialization of Buildings and Agents}
\subsection{Social forces}
\subsubsection{Architecture forces}
\subsubsection{Exit forces}
\subsubsection{Agent forces}
\subsection{Physical forces}
\subsubsection{Wall forces}
\subsubsection{Agent forces}
\subsection{Walking speed}
\subsection{Flooding}


\section{Simulation Results and Discussion}
\subsection{Simple evacuation bottleneck: One exit}
\subsection{Simple evacuation bottleneck: Two exits}
\subsection{Evacuation through a road network}
\subsection{Evacuation through a road network with topography and flooding}
\subsection{Evacuation of a beach in the case of a tsunami event}

\section{Summary and Outlook}

\section{References}






\end{document}  



 
